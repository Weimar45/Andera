% Options for packages loaded elsewhere
\PassOptionsToPackage{unicode}{hyperref}
\PassOptionsToPackage{hyphens}{url}
\PassOptionsToPackage{dvipsnames,svgnames,x11names}{xcolor}
%
\documentclass[
  letterpaper,
  DIV=11,
  numbers=noendperiod]{scrartcl}

\usepackage{amsmath,amssymb}
\usepackage{lmodern}
\usepackage{iftex}
\ifPDFTeX
  \usepackage[T1]{fontenc}
  \usepackage[utf8]{inputenc}
  \usepackage{textcomp} % provide euro and other symbols
\else % if luatex or xetex
  \usepackage{unicode-math}
  \defaultfontfeatures{Scale=MatchLowercase}
  \defaultfontfeatures[\rmfamily]{Ligatures=TeX,Scale=1}
\fi
% Use upquote if available, for straight quotes in verbatim environments
\IfFileExists{upquote.sty}{\usepackage{upquote}}{}
\IfFileExists{microtype.sty}{% use microtype if available
  \usepackage[]{microtype}
  \UseMicrotypeSet[protrusion]{basicmath} % disable protrusion for tt fonts
}{}
\makeatletter
\@ifundefined{KOMAClassName}{% if non-KOMA class
  \IfFileExists{parskip.sty}{%
    \usepackage{parskip}
  }{% else
    \setlength{\parindent}{0pt}
    \setlength{\parskip}{6pt plus 2pt minus 1pt}}
}{% if KOMA class
  \KOMAoptions{parskip=half}}
\makeatother
\usepackage{xcolor}
\setlength{\emergencystretch}{3em} % prevent overfull lines
\setcounter{secnumdepth}{-\maxdimen} % remove section numbering
% Make \paragraph and \subparagraph free-standing
\ifx\paragraph\undefined\else
  \let\oldparagraph\paragraph
  \renewcommand{\paragraph}[1]{\oldparagraph{#1}\mbox{}}
\fi
\ifx\subparagraph\undefined\else
  \let\oldsubparagraph\subparagraph
  \renewcommand{\subparagraph}[1]{\oldsubparagraph{#1}\mbox{}}
\fi


\providecommand{\tightlist}{%
  \setlength{\itemsep}{0pt}\setlength{\parskip}{0pt}}\usepackage{longtable,booktabs,array}
\usepackage{calc} % for calculating minipage widths
% Correct order of tables after \paragraph or \subparagraph
\usepackage{etoolbox}
\makeatletter
\patchcmd\longtable{\par}{\if@noskipsec\mbox{}\fi\par}{}{}
\makeatother
% Allow footnotes in longtable head/foot
\IfFileExists{footnotehyper.sty}{\usepackage{footnotehyper}}{\usepackage{footnote}}
\makesavenoteenv{longtable}
\usepackage{graphicx}
\makeatletter
\def\maxwidth{\ifdim\Gin@nat@width>\linewidth\linewidth\else\Gin@nat@width\fi}
\def\maxheight{\ifdim\Gin@nat@height>\textheight\textheight\else\Gin@nat@height\fi}
\makeatother
% Scale images if necessary, so that they will not overflow the page
% margins by default, and it is still possible to overwrite the defaults
% using explicit options in \includegraphics[width, height, ...]{}
\setkeys{Gin}{width=\maxwidth,height=\maxheight,keepaspectratio}
% Set default figure placement to htbp
\makeatletter
\def\fps@figure{htbp}
\makeatother

\KOMAoption{captions}{tableheading}
\makeatletter
\makeatother
\makeatletter
\makeatother
\makeatletter
\@ifpackageloaded{caption}{}{\usepackage{caption}}
\AtBeginDocument{%
\ifdefined\contentsname
  \renewcommand*\contentsname{Table of contents}
\else
  \newcommand\contentsname{Table of contents}
\fi
\ifdefined\listfigurename
  \renewcommand*\listfigurename{List of Figures}
\else
  \newcommand\listfigurename{List of Figures}
\fi
\ifdefined\listtablename
  \renewcommand*\listtablename{List of Tables}
\else
  \newcommand\listtablename{List of Tables}
\fi
\ifdefined\figurename
  \renewcommand*\figurename{Figure}
\else
  \newcommand\figurename{Figure}
\fi
\ifdefined\tablename
  \renewcommand*\tablename{Table}
\else
  \newcommand\tablename{Table}
\fi
}
\@ifpackageloaded{float}{}{\usepackage{float}}
\floatstyle{ruled}
\@ifundefined{c@chapter}{\newfloat{codelisting}{h}{lop}}{\newfloat{codelisting}{h}{lop}[chapter]}
\floatname{codelisting}{Listing}
\newcommand*\listoflistings{\listof{codelisting}{List of Listings}}
\makeatother
\makeatletter
\@ifpackageloaded{caption}{}{\usepackage{caption}}
\@ifpackageloaded{subcaption}{}{\usepackage{subcaption}}
\makeatother
\makeatletter
\makeatother
\ifLuaTeX
  \usepackage{selnolig}  % disable illegal ligatures
\fi
\IfFileExists{bookmark.sty}{\usepackage{bookmark}}{\usepackage{hyperref}}
\IfFileExists{xurl.sty}{\usepackage{xurl}}{} % add URL line breaks if available
\urlstyle{same} % disable monospaced font for URLs
\hypersetup{
  pdftitle={Archivo de Consultas de Filtrado sobre Análisis Metagenómicos},
  pdfauthor={Alejandro Navas González},
  colorlinks=true,
  linkcolor={blue},
  filecolor={Maroon},
  citecolor={Blue},
  urlcolor={Blue},
  pdfcreator={LaTeX via pandoc}}

\title{\textbf{Archivo de Consultas de Filtrado sobre Análisis
Metagenómicos}}
\author{\textbf{Alejandro Navas González}}
\date{6/15/23}

\begin{document}
\maketitle
\textbf{1. ¿Qué significa el filtrado por abundancia total de las OTUs?}

La abundancia total de una OTU (Unidad Taxonómica Operativa) en todas
las muestras se refiere a la suma de las cuentas (o lecturas) de esa OTU
en todas las muestras. Por ejemplo, si se dispone de tres muestras y una
OTU tiene 2 cuentas en la primera muestra, 1 cuenta en la segunda
muestra y 2 cuentas en la tercera muestra, la abundancia total de esa
OTU en todas las muestras sería 2 + 1 + 2 = 5. Esto es diferente del
umbral de presencia en un número de muestras determinado. Es decir, una
OTUs puede tener una abundancia total de 10 y estar solamente presente
en una muestra.En este contexto del análisis de microbiomas, una OTU con
una abundancia total baja podría ser una especie rara en la comunidad, o
podría ser un artefacto de secuenciación.

\textbf{2. ¿Cuáles son los métodos de filtrado más comunes y en qué
radica su importancia?}

\begin{itemize}
\item
  \textbf{Filtrado por abundancia}: Este tipo de filtrado implica
  eliminar las OTUs que tienen una abundancia total (es decir, la suma
  de las cuentas en todas las muestras) por debajo de un cierto umbral.
  Este tipo de filtrado puede ser relevante para eliminar las OTUs raras
  o los posibles artefactos de secuenciación, que podrían añadir ruido a
  los análisis.
\item
  \textbf{Filtrado por presencia en muestras}: Este tipo de filtrado
  implica eliminar las OTUs que están presentes en menos de un cierto
  número de muestras. Este tipo de filtrado puede ser relevante para
  eliminar las OTUs que son raras o que podrían ser artefactos de
  secuenciación.
\item
  \textbf{Filtrado por taxonomía}: Este tipo de filtrado implica
  eliminar las OTUs que pertenecen a ciertos grupos taxonómicos. Este
  tipo de filtrado puede ser relevante, por ejemplo, para eliminar las
  OTUs de la clase de los cloroplastos o la familia de las mitocondrias,
  que son organismos que no son bacterias y que pueden estar presentes
  en los datos debido a la contaminación durante el proceso de
  secuenciación. También puede ser relevante para eliminar una especie
  testigo que se haya añadido al experimento para controlar la calidad
  del proceso de secuenciación.
\item
  \textbf{Filtrado por variabilidad}: Este tipo de filtrado implica
  eliminar las OTUs que tienen una variabilidad baja o alta en su
  abundancia entre las muestras. Este tipo de filtrado puede ser
  relevante para eliminar las OTUs que son constantes o muy variables
  entre las muestras, ya que podrían no ser informativas para los
  análisis posteriores.
\end{itemize}

\textbf{3. ¿Qué sucede si en el filtrado por taxonomía se hallan
especies del filo \emph{Cyanobacteria} con la clase \emph{Chloroplast}?}

En los estudios de microbiomas, la presencia de secuencias asignadas al
filo \textbf{\emph{Cyanobacteria}} y a la clase
\textbf{\emph{Chloroplast}} puede ser indicativa de contaminación por
ADN de origen no bacteriano. Los cloroplastos, que son los orgánulos
responsables de la fotosíntesis en las plantas y las algas, son en
realidad descendientes de cianobacterias que fueron incorporadas por una
célula eucariota por un proceso llamado endosimbiosis. Como resultado,
los cloroplastos comparten muchas similitudes genéticas con las
cianobacterias, y las secuencias de ADN de los cloroplastos pueden ser
erróneamente asignadas a las cianobacterias en los análisis de
microbiomas. Por ende, si se observa una gran cantidad de secuencias
asignadas a cloroplasto en los datos de la secuenciación, hay varias
posibilidades:

\begin{itemize}
\item
  \textbf{Contaminación durante la recolección de muestras}: Si las
  muestras fueron recolectadas de un ambiente donde las plantas o las
  algas son abundantes, es posible que el ADN de los cloroplastos haya
  contaminado las muestras.
\item
  \textbf{Contaminación durante la extracción de ADN o la
  secuenciación}: Si las muestras fueron procesadas en un laboratorio
  donde también se estaban manejando plantas o algas, es posible que el
  ADN de los cloroplastos haya contaminado las muestras o los reactivos.
\item
  \textbf{Contaminación en la base de datos de referencia}: Si la base
  de datos de referencia que estás utilizando para asignar las
  secuencias a los taxones contiene secuencias de cloroplastos que están
  mal etiquetadas como cianobacterias, esto podría llevar a una
  asignación errónea de las secuencias.
\end{itemize}

Ahora bien, se ofrece una opción alternativa y es que las secuencias
asignadas a la clase \emph{Chloroplast} sean de hecho procedentes de
cianobacterias. Las cianobacterias son un grupo diverso de bacterias
fotosintéticas que se encuentran en una variedad de ambientes,
incluyendo el agua dulce, el agua de mar, el suelo y algunas condiciones
extremas como los manantiales termales y las zonas áridas. Si las
muestras provienen de un ambiente de este calado, donde las
cianobacterias son comunes, es posible que las secuencias que se
observan sean realmente de cianobacterias y no de cloroplastos. En este
caso, no interesa filtrarlas. Para determinar si las secuencias son
realmente de cianobacterias, se pueden considerar varias estrategias:

\begin{itemize}
\item
  \textbf{Revisar la asignación taxonómica a niveles más bajos}: Si las
  secuencias están asignadas a géneros o especies que son conocidos por
  ser cianobacterias, esto podría indicar que son realmente de
  cianobacterias.
\item
  \textbf{Revisar la literatura y los datos existentes}: Si otros
  estudios han encontrado cianobacterias en el mismo tipo de muestras o
  en el mismo ambiente, esto podría apoyar la idea de que las secuencias
  son realmente de cianobacterias.
\item
  \textbf{Realizar análisis filogenéticos}: Se podría construir un árbol
  filogenético con las secuencias y compararlo con árboles de referencia
  para ver si las secuencias se agrupan con las cianobacterias
  conocidas.
\end{itemize}

En última instancia, la interpretación de estos datos requerirá un
juicio basado en el conocimiento del ambiente de muestreo, el proceso de
secuenciación y análisis, y la biología de las cianobacterias y los
cloroplastos.

\textbf{4. ¿Qué significa el uso de \emph{Aliivibrio fischeri} como
testigo en el filtrado por taxonomía?}

\textbf{\emph{Aliivibrio fischeri}} es una especie de bacteria
gram-negativa que se encuentra comúnmente en ambientes marinos. A.
fischeri es conocida por su capacidad para producir luz, un fenómeno
conocido como bioluminiscencia. Esta bacteria forma una relación
simbiótica con varios animales marinos, como el calamar hawaiano
Euprymna scolopes, donde la bacteria coloniza un órgano especializado en
el calamar y produce luz que el calamar utiliza para camuflarse. A.
fischeri se utiliza a menudo como organismo modelo en la investigación
de la bioluminiscencia y la simbiosis.

En el contexto de los análisis de microbiomas, un \textbf{organismo
testigo}, también conocido como \textbf{control positivo}, es un
organismo que se añade intencionadamente a las muestras o al proceso de
secuenciación para controlar la calidad del experimento. Por ejemplo, se
puede añadir una cantidad conocida de \emph{A. fischeri} a tus muestras
antes de la extracción de ADN. Luego, después de la secuenciación, se es
capaz de detectar A. fischeri en los datos de secuenciación. Si no se
pudiera detectar A. fischeri, o si la abundancia de A. fischeri fuera
muy diferente de la cantidad añadida, esto podría indicar un problema
con la extracción de ADN o la secuenciación. Además, al conocer la
cantidad exacta de A. fischeri que se ha añadido, es posible utilizar
este organismo testigo para calibrar los datos de abundancia. Esto puede
ser útil para convertir las cuentas de lecturas de secuenciación, que
pueden ser afectadas por factores como la eficiencia de la extracción de
ADN y la profundidad de secuenciación, en estimaciones más precisas de
la abundancia de los organismos en tus muestras.

Por lo tanto, el uso de un organismo testigo como \emph{A. fischeri}
puede ser una herramienta valiosa para controlar la calidad de los
experimentos y mejorar la precisión de los análisis de microbiomas.



\end{document}
